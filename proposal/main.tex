\documentclass[11pt]{article}
\usepackage{acl}
\usepackage{times}
\usepackage{latexsym}
\usepackage{amsmath}
\usepackage{graphicx}
\usepackage{hyperref}
\usepackage[T1]{fontenc}
\usepackage[utf8]{inputenc}
\usepackage{microtype}
\usepackage{booktabs}
\usepackage{listings}
\usepackage{subcaption}
\usepackage{float}
\begin{document}

\title{Ecs251 project proposal}

\author{
  Harshil Patel$^{1}$ \ , Anugya Sharma$^{2}$
  \\
  $^{1}$Department of Computer Science, University of California, Davis \\
}

\maketitle

\section{Introduction \& Motivation}
The advancements in multicore architecture and multiprocessor computers are increasing
 and ongoing[14, 15, 16]. To keep up with this trend, we need techniques like 
 multithreading  to maximize the performance of  ever-advancing cpu’s[13]. 
 Multithreading is of vital importance for this because it can harness the power of multiprocessor 
 computer, improve system reliability by preventing one operation from negatively 
 affecting another in a program(user interface event affecting time-critical operation),
  and maximize cpu use (in programs that read/write from a file, perform I/O, or poll 
 the user interface)[13].

While a very efficient tool for harnessing the power of multiprocessor computer, 
multithreading does come with some challenges. The top most important challenge in 
multithreading is the management of thread pools[11], namely the size of the thread 
pool. The number of threads in a thread pool determine  the changes in response times 
and resource utilization [11]. In our research, we have come up with a “dynamic thread 
pool” with varying number of threads such that we can optimize minimum response time for
 maximum resource utilization. 

example citation \cite{hu2024fine}.

\bibliographystyle{acl_natbib}
\bibliography{references}
\end{document}
