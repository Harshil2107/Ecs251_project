\documentclass[11pt]{article}
\usepackage{acl}
\usepackage{times}
\usepackage{latexsym}
\usepackage{amsmath}
\usepackage{graphicx}
\usepackage{hyperref}
\usepackage[T1]{fontenc}
\usepackage[utf8]{inputenc}
\usepackage{microtype}
\usepackage{booktabs}
\usepackage{listings}
\usepackage{subcaption}
\usepackage{float}

\begin{document}

\title{Ecs251 project proposal}

\author{
  Harshil Patel$^{1}$ \ , Anugya Sharma$^{1}$
  \\
  $^{1}$Department of Computer Science, University of California, Davis \\
}

\maketitle

\section{Introduction \& Motivation}
The advancements in multicore architecture and multiprocessor computers are increasing
 and ongoing \cite{adam2022co}. To keep up with this trend, we need techniques like 
 multithreading  to maximize the performance of  ever-advancing cpu’s (ni manula). 
 Multithreading is of vital importance for this because it can harness the power of multiprocessor 
 computer, improve system reliability by preventing one operation from negatively 
 affecting another in a program(user interface event affecting time-critical operation),
  and maximize cpu use (in programs that read/write from a file, perform I/O, or poll 
 the user interface)[13]

While a very efficient tool for harnessing the power of multiprocessor computer, 
multithreading does come with some challenges. The top most important challenge in 
multithreading is the management of thread pools\cite{lee2011novel}, namely the size of the thread 
pool. The number of threads in a thread pool determine  the changes in response times 
and resource utilization [11]. In our research, we have come up with a “dynamic thread 
pool” with varying number of threads such that we can optimize minimum response time for
 maximum resource utilization. 

 Start Rough note for Q2 ....................Current heuristics that are used to
  determine the number of threads are flawed[4]. There has been some researches 
  on a dynamic model for thread size. In the beginning, they just had the watermark 
  model(adjusts the number of threads according to the current number of incoming 
  requests)[20] .While the watermark model solves the problem of “efficient usage of
   resources”, it doesn’t efficiently obtain the required number of threads in 
   advance[19]. There are some prediction based models, but In the prediction based
    models, only factor such as the request rate or number of worker threads is 
    considered in the management. [19,20,11] One research that has proposed dynamic
     thread pool(prediction based) model, themselves point out that the model lacks 
     testing for a longer period of time and on general server environment which provides
      various services simultaneously[4].Trendy exponential moving average model 
      proposed in one study doesn’t have the desired accuracy in its predictions[11]. 

 All these researches into dynamic sizes of pool, but none look at the memory usage/CPU
  overheads and whether that are significantly lower than static pool to signify the 
  performance cost, whatever they may be. 
 We have to look at this , so that this informs the people making decision to choose 
 dynamic thread pools moving forward. ………………………End Rough note for Q2

example citation \cite{hu2024fine}.

\bibliographystyle{acl_natbib}
\bibliography{references}
\end{document}
