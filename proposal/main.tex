\documentclass[11pt]{article}
\usepackage{acl}
\usepackage{times}
\usepackage{latexsym}
\usepackage{amsmath}
\usepackage{graphicx}
\usepackage{hyperref}
\usepackage[T1]{fontenc}
\usepackage[utf8]{inputenc}
\usepackage{microtype}
\usepackage{booktabs}
\usepackage{listings}
\usepackage{subcaption}
\usepackage{float}

\begin{document}

\title{Ecs251 project proposal}

\author{
  Harshil Patel$^{1}$ \ , Anugya Sharma$^{1}$
  \\
  $^{1}$Department of Computer Science, University of California, Davis \\
}

\maketitle

\section{Introduction \& Motivation}
The advancements in multicore architecture and multiprocessor computers are increasing
 and ongoing \cite{adam2022co}. To keep up with this trend, we need techniques like 
 multithreading  to maximize the performance of  ever-advancing cpu’s \cite{ni}. 
 Multithreading is of vital importance for this because it can harness the power of multiprocessor 
 computer, improve system reliability by preventing one operation from negatively 
 affecting another in a program(user interface event affecting time-critical operation),
  and maximize cpu use (in programs that read/write from a file, perform I/O, or poll 
 the user interface) \cite{ni}.

While a very efficient tool for harnessing the power of multiprocessor computer, 
multithreading does come with some challenges. The top most important challenge in 
multithreading is the management of thread pools \cite{lee2011novel}, namely the size of the thread 
pool. The number of threads in a thread pool determine  the changes in response times 
and resource utilization \cite{lee2011novel}. In our research, we aim to collect empirical data that informs
decisions on thread pool size. In doing so, we also aim to propose a “dynamic thread 
pool” with varying number of threads such that we can optimize minimum response time for
 maximum resource utilization. We will evaluate evaluate the performance trade-offs between 
 static and dynamic thread pools under varying workloads and system configurations and answer 
 the question: At what point does the overhead of dynamic thread creation outweigh its memory 
 savings compared to static thread pools?

 \section{Background \& Related Work}

 Current heuristics that are used to
  determine the number of threads are flawed \cite{ling2000analysis}. In our preliminary research, we 
  find that there have been few researches on dynamic model for thread size. One of the first such
   model was the Watermark model which adjusts the number of threads according to the current number of incoming 
  requests \cite{kim2007prediction}. The issue with this model is that while it solves the problem of “efficient usage of
   resources”, it doesn’t efficiently obtain the required number of threads in 
   advance \cite{kang2008prediction}. Furthermore, there are also prediction based models to determine thread pool size, but in such
    models, only factors such as the request rate or number of worker threads is 
    considered in the management \cite{lee2011novel}. One of the research that has proposed dynamic
     thread pool model(prediction based), themselves point out that the model lacks 
     testing for a longer period of time and on general server environment which provides
      various services simultaneously \cite{ling2000analysis}. Another research adds to the prediction model by
       proposing what they call the "Trendy exponential moving average model" and again 
       the issue with this model seems to be that it doesn't have the desired accuracy in its prediction
       \cite{lee2011novel}. 

 Of the many researches into dynamic sizes of pool, none look at the memory usage/CPU
  overheads and whether those are significantly lower than static pool to signify the 
  performance cost. The trade off for replacing static threads affect the performance and 
  costs of cloud-based systems \cite{freire2021performance}, use of dynamic pools in web services
  handling could have latency spikes during high traffic. Not just costs and such latency issues, there is also the question 
  of system design such as database connection pools, task scheduling in distributed systems, etc which
   all use thread pools to manage concurrency. There is a lack of comprehensive studies that quantify the
    performance implications of static vs dynamic thread pools across different 
    workloads and system configurations. By systematically evaluating these trade-offs,
     we can provide practical guidelines for system architects to choose the right thread
      pool strategy based on their specific workload and performance requirements. 

example citation \cite{hu2024fine}.

\bibliographystyle{acl_natbib}
\bibliography{references}
\end{document}
